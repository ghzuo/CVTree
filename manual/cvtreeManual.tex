\documentclass[a4paper,12pt]{article}
\usepackage{amssymb}
\usepackage{graphicx}
\usepackage{amsmath}
\begin{document}

\title{CVTree \\\Large{(Standalone Version)}\\User's Manual}
\author{Guanghong Zuo and Bailin HAO}
\date{\today}
\maketitle

%\newpage

CVTree stands for {\bf Composition Vector Tree} which is the
implementation of an alignment-free algorithm to generate a
dissimilarity matrix from comparatively large collection of DNA or Amino
Acid sequences, preferably whole-genome data, for phylogenetic studies.

There are two versions of the program:
\begin{enumerate}\itemsep 0pt
\item {\bf CVTree Web Server} which has been published twice in the Web
  Server Issues of {\it Nucleic Acids Research}, \cite{qlh04}
  and\\ \cite{xh09}. The lastest CVTree Web Server, CVTree3 Web Server \cite{zh15}, have two identical but independent installations:
  \begin{itemize}
  	\item Fudan University, Shanghai: \\
  	  {\tt http://tlife.fudan.edu.cn/cvtree3/} 
  	\item Beijing Institute of Genomics, Beijing:\\
  	{\tt http://cvtree.cn \\ http://bigd.big.ac.cn/cvtree} 
  \end{itemize}

\item {\bf CVTree Standalone Version} which is provided to those who are
  interested in the intermediate results, e.g., the collection of all
  CVs, or deal with extremely huge datasets of their own. We provide
  also a few options and scripts that were not available in the Web
  Server versions. 
\end{enumerate}

\section{The Algorithm}

The algorithm of CVTree consists of the following steps:
\begin{enumerate}\itemsep 0pt
\item Fix a string length $K$ ($K\in [3,7]$ for Amino Acid sequences and
  $K\in [3,16]$ for nucleotide sequences in 32-bit system). Read in the
  sequence collection of each species separately. Count the number of
  all $K$, $K-1$ and $K-2$ tuples for a species. A {\it raw} Composition
  Vector (CV) of dimension $4^K$ or ${20}^K$ is formed by putting the
  counts of $K$-tuples in lexicographic order.
  \item Calculate the subtraction score for the i-th $K$-tuple:
\begin{equation*}
  a_i(a_1a_2 \cdots a_K) \equiv \frac{f(a_1a_2 \cdots a_K) -
    f^0(a_1a_2 \cdots a_K)}{f^0(a_1a_2 \cdots a_K)}
\end{equation*}
where $f(a_1a_2 \cdots a_K)$ is the frequency of $K$-tuple, $f^0(a_1a_2
\cdots a_K)$ is the frequency predicted from that of $(K-1)$ and $(K-2)$
tuples by using a $(K-2)$-th Markov assumption, \cite{qwh04}.  All
components of the {\it raw} CV is replaced by its subtraction score to
yield a renormalized CV.
\item Using the renormalized CVs to calculate the pairwise dissimilarity
  between two species:
  $$d(A, B)=(1-C(\vec{CV_A},\vec{CV_B}))/2,$$
where
\begin{equation*}
  C(\vec{CV_A},\vec{CV_B})=\frac{\sum_{i=1}^NA_i \times
    B_i}{(\sum_{i=1}^NA_i^2 \times \sum_{i=1}^NB_i^2)^{\frac{1}{2}}}
\end{equation*}

\item Then obtain the phylogenetic tree (Newick Format) based on this
  dissimilarity matrix by Neighbor Joint method.
\end{enumerate}
For more detailed description of the algorithm please consult
\cite{qwh04}. 



\section{The Installation}
\subsection{Preparation}
\begin{itemize}
	\item cmake $\geq$ 2.6
  \item g++ $\geq$ 4.8 or other compiler supporting C++11 standard
  \item compiler with support openmp for parallel
	\item Library: libz, netcdf, netcdf\_cpp
\end{itemize}

\subsection{Compiling}
Unzip or checkout the source files, Obtained the compiling option by
cmake
\begin{itemize}
	\item unzip the package file and change into it
	\item mkdir build and change into it
	\item cmake ../src or add some options you wanted, e.g.:\\ -DCMAKE\_INSTALL\_PREFIX=/usr/local
	\item make
	\item make install
\end{itemize}

\section{Programs and Command-Line Options}

The main program was implemented in C++. For most purposes, the C++
program \verb!cvtree! is enough for the end user. However we supply some
Perl scripts to treat extremely massive input data (e.g., exceeding
several gigabytes). If you encounter ``Out of memory'' warning when
running CVTree program by \verb!-o! or \verb!-l! option, you can try to
use \verb!-c! option instead. Option \verb!-c! will output separated CV
files into the given directory, which can be used by \verb!bdist!(or
\verb!batch_dist.pl!) to calculate the final dissimilarity matrix.

\begin{enumerate}\itemsep 0pt
  \item cv -- Generate CV files from input data
\begin{verbatim}
cv [ -I faa ]          input genome file directory, defaut: faa
   [ -i list ]         input species list, defaut: list
   [ -k '3 4 5 6 7' ]  values of k, defaut: N = 3 4 5 6 7
   [ -g faa ]          the type of genome file, defaut: faa
   [ -O cv ]           output cv directory, defaut: cv
   [ -S 0/1]           whethe do the subtract, defaut: 1
   [ -h ]              disply this information
\end{verbatim}

  \item tree -- Generate newick tree based on CV files
\begin{verbatim}
tree
 [ -o dist.matrix ]   Output distance matrix, 
                      defaut: dist.matrix
 [ -I extdir ]        Directory of extend cv files, 
                      defaut: cv
 [ -i infile ]        Extend cv file list, 
                      defaut: no extend cv used
 [ -s cv6.gz ]        Suffix of cv file, 
                      defaut: cv6.gz
 [ -E orgdir  ]       Directory of the orginal cv files
 [ -m indist.matrix ] Input distance matrix, 
                      default: no input matrix used
 [ -e orglist ]       Name of selected genomes, 
                      which are in input distance matrix
 [ -n ndxlist ]       Index of selected genomes, 
                      which are input distance matrix
                      The index file used first!
 [ -t taxfile ]       input taxonomy information
 [ -T ]               Do not output taxonomy information
 [ -M <N> ]           Runing memory size as G roughly, 
                      default 80% physical memory
 [ -C ]               Force use the netcdf compress distance matrix
 [ -h ]               disply this information
\end{verbatim}
\item cvdump -- convect the binary cv file to acsii file
\begin{verbatim}
cvdump  -i <cvfile>        input file name
 [ -g faa|ffn ]      the type of genome file, defaut: faa
 [ -n ]              output the number code, default: the letters
 [ -h ]              disply this information
\end{verbatim}

  \item runCVTree.sh (in example folder) -- Easy script to obtain the result.
\end{enumerate}

\section{Citing CVTree in a Publication}

Please cite:
\begin{enumerate}\itemsep 0pt
\item Ji Qi, Bin Wang, Bailin Hao (2004), Whole proteome prokaryote
  phylogeny without sequence alignment: a K-string composition approach,
  {\it Journal of Molecular Evolution}, {\bf 58}: 1 -- 11.
\item Guanghong Zuo and Bailin Hao (2015) CVTree3 Web Server for Whole-genome-based and Alignment-free Prokaryotic Phylogeny and Taxonomy.
\newblock {\em Genomics Proteomics Bioinformatics, } {\bf 13}: 321--331.
\end{enumerate}


\section{Version History and Contributors}

Since 2002 the CVTree software has undergone many revisions. A few major
versions were:
\begin{enumerate}\itemsep 0pt
\item Web Server CVTree 1.0 was written by Ji Qi, Hong Luo and Bailin Hao
\item Most 3.x versions of Standalone CVTree was written by Lei Gao;
  Ver. 3.9.6 was written by Ji Qi.
\item Web Server CVTree 2.0 was written by Zhao Xu and Bailin Hao
\item Standalone CVTree 4.4 was written by Zhao Xu
\item Web Server CVTree 3.0 was written by Guanghong Zuo and Bailin Hao
\item Standalone CVTree 5.0 was written by Guanghong Zuo
\end{enumerate}

\begin{thebibliography}{}
\bibitem[Qi {\em et~al.}, 2004{\em{a}}]{qlh04}
Qi,J., Luo,H.  and Hao,B. (2004{\em{a}}) Cvtree: a phylogenetic tree
  reconstruction tool based on whole genomes.
\newblock {\em Nucleic acids research, } {\bf 32}, W45--7.
\newblock PMID: 15215347.

\bibitem[Qi {\em et~al.}, 2004{\em{b}}]{qwh04}
Qi,J., Wang,B.  and Hao,B. (2004{\em{b}}) Whole proteome prokaryote phylogeny
  without sequence alignment: a k-string composition approach.
\newblock {\em Journal of Molecular Evolution, } {\bf 58}, 1--11.

\bibitem[Xu and Hao, 2009]{xh09}
Xu,Z. and Hao,B. (2009{\em{}}) Cvtree update: a newly designed phylogenetic
  study platform using composition vectors and whole genomes.
\newblock {\em Nucleic Acids Research, } {\bf 37 Web Server Issue}, W174--W178.
\bibitem[Zuo and Hao, 2015]{zh15}
Zuo,G. and Hao,B. (2015{\em{}}) CVTree3 Web Server for Whole-genome-based and Alignment-free Prokaryotic Phylogeny and Taxonomy.
\newblock {\em Genomics Proteomics Bioinformatics, } {\bf 13}, 321--331.
\end{thebibliography}
\end{document}
